\begin{appendices}

\section{Workload}

In the initial week our combined spent hours was only around 14 hours. But the 3 weeks after we've been working for this project non-stop and easily spent 110 hours each. In the end, we both worked around 120 hours each on this project.
\begin{itemize}
	\item \textbf{Data filtering: 60\%.} Most of our time was spent filtering data. There were quite a lot of inconsistencies, confusion about what was safe to delete and what was mandatory for the project and what the influence would be of the decisions we made. Many iterations happened before we reached a result we were satisfied with.
	\item \textbf{Truth generation: 5\%.} This was generally easy, though we were held up for a while since we came to the realization that there is no exact solution: Depending on interpretation, the desired endpath could be different for a given path. This has been explained in our project.
	\item \textbf{Splitting training/test data: 5\%.} Using the modules from the \textbf{scikit} package, this was generally easy to implement. Of course, formatting results and code in a readable way still takes time.
	\item \textbf{The model: 20\%.} We spent time creating a graph representation using transition matrices, then we realized that such approach was not optimal, because the large amount of zeros against the little amount of real probabilities; a graph representation with nodes and edges was the solution to reduce useless spaces. Then we also tried to use the library hmmlearn to get the prediction, however we realised that was outside of the scope of the project and we using our graph representation as a Markov chain and we implemented a search method to get our deepest path according with a confident interval.
	\item \textbf{Report: 10\%} We are probably being modest when we say the report \textit{only} took 10\% of our time. Writing comprehensible, informative and compact text requires time.
\end{itemize} 

\end{appendices}